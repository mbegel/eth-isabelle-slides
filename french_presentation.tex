\documentclass{beamer}

\usepackage{mathtools}
\usepackage{pgf}
\usepackage{tikz}
\usepackage[normalem]{ulem}
\usetheme{Data61}
\renewcommand{\t}[1]{\texttt{#1}}
\usepackage[frenchb]{babel}


\usepackage{amsmath}
\usepackage{amsfonts}
\usepackage{amssymb}
\usepackage{mathtools}
%\usepackage[usenames,dvipsnames,svgnames,table]{xcolor}
\usepackage{caption}

%%Found on Internet for Isabelle
% needed packages
\usepackage{verbatim}
\usepackage{pstricks,pst-node}
\usepackage{fancyvrb}
%\usepackage[fancyvrb]{listings}
\usepackage{listings}
% X-Symbols special characters
\DeclareTextSymbol{\textbackslash}{T1}{92}
\newcommand{\nsubset}{\not\subset}
%\newcommand{\textflorin}{\textit{f}}
\newcommand{\setB}{{\mathord{\mathbb B}}}
\newcommand{\setC}{{\mathord{\mathbb C}}}
\newcommand{\setN}{{\mathord{\mathbb N}}}
\newcommand{\setQ}{{\mathord{\mathbb Q}}}
\newcommand{\setR}{{\mathord{\mathbb R}}}
\newcommand{\setZ}{{\mathord{\mathbb Z}}}
\newcommand{\coloncolon}{\mathrel{::}}
\newcommand{\lsemantics}{\mathopen{\lbrack\mkern-3mu\lbrack}}
\newcommand{\rsemantics}{\mathclose{\rbrack\mkern-3mu\rbrack}}
\newcommand{\lcata}{\mathopen{(\mkern-3mu\mid}}
\newcommand{\rcata}{\mathclose{\mid\mkern-3mu)}}

% new language
\lstdefinelanguage{Isar}%
  {keywords={
        _first,theory,end,types,datatype,consts,defs,primrec,
        syntax,translations,apply,definition,fun,function, lemma,theorem,corollary,
        done,sorry,goal,proof,fixed variables,_last, inductive, where, if, then, else, where},
    keywordstyle=\color{Data61 light forest}\bfseries,
    keywordstyle=[3]\itshape,
        sensitive=true,
        morecomment=[l]{-- },
        literate=
                {\\<not>}{{$¬$}}1 {\\<times>}{{$×$}}1                 
                {\\<Rightarrow>}{{$\Rightarrow$}}1%
                {\\<equiv>}{{$\equiv$}}1 {~=}{{$\not=$}}1                         
                {\\<rightleftharpoons>}
                        {{$\rightleftharpoons$}}1
                {\\<exists>}{{$\exists$}}1
                {\\<parallel>}{{$\parallel$}}1
                {\\<Longrightarrow>}
                        {{$\Longrightarrow$}}1
                 {\\<lbrakk>}{{$\lsemantics$}}1
                {\\<rbrakk>}{{$\rsemantics$}}1
                {\\<longrightarrow>}
                        {{$\longrightarrow$}}1
                {\\<and>}{{$\land$}}1
                {\\<or>}{{$\lor$}}1
                {\\<bigwedge>}{{$\bigwedge$}}1
                {\\<forall>}{{$\forall$}}1
                {\\<langle>}{{$\langle$}}1
                {\\<rangle>}{{$\rangle$}}1
                {\\<models>}{{$\models$}}1
                {ü}{"u}1 {ä}{"a}1 {ö}{"o}1
                {Ü}{"U}1 {Ä}{"A}1 {Ö}{"O}1 {ß}{{ss}}1
  }[keywords,comments,strings]%

% useful defaults
\lstset{
  basicstyle=\footnotesize,
  commentstyle=\rm\it,
  flexiblecolumns=false,
  breaklines=true,
  breakautoindent=false
}

% possible usage
%\lstinline!apply(simp)!
%\begin{lstlisting}[language=Isar]{}
%datatype type = IntType | BoolType | FunType type type | ErrorType
%\end{lstlisting}
%\lstinputlisting[language=Isar]{foo.thy}

% Taken from Lena Herrmann at 
% http://lenaherrmann.net/2010/05/20/javascript-syntax-highlighting-in-the-latex-listings-package
\definecolor{lightgray}{rgb}{.9,.9,.9}
\definecolor{darkgray}{rgb}{.4,.4,.4}
\definecolor{purple}{rgb}{0.65, 0.12, 0.82}

\lstdefinelanguage{JavaScript}{
  keywords={typeof, new, true, false, catch, function, return, null, catch, switch, var, if, in, while, do, else, case, break, returns, contract},
  keywordstyle=\color{blue}\bfseries,
  ndkeywords={class, export, boolean, throw, implements, import, this},
  ndkeywordstyle=\color{darkgray}\bfseries,
  identifierstyle=\color{black},
  sensitive=false,
  comment=[l]{//},
  morecomment=[s]{/*}{*/},
  commentstyle=\color{purple}\ttfamily,
  stringstyle=\color{red}\ttfamily,
  morestring=[b]',
  morestring=[b]"
}

\lstset{
   extendedchars=true,
   tabsize=2,
   breaklines=true,
   captionpos=b
}


%%From Guillaume

\definecolor{pblue}{rgb}{0.13,0.13,1}
\definecolor{pgreen}{rgb}{0,0.5,0}
\definecolor{pred}{rgb}{0.9,0,0}
\definecolor{pgrey}{rgb}{0.46,0.45,0.48}

\lstset{%language=Java,
        showspaces=false,
        showtabs=false,
        breaklines=true,
        showstringspaces=false,
        breakatwhitespace=true,
%        commentstyle=\color{pgreen},
%       keywordstyle=\color{pblue},
%        stringstyle=\color{pred},
        basicstyle=\ttfamily,
%        stepnumber=1,
%        firstnumber=1,
%        linewidth=0.6\textwidth,
%        xleftmargin=0.2\textwidth,
%        xrightmargin=0.2\textwidth,
        numberfirstline=false}

\title{Logique de programme pour le bytecode d'Ethereum}
\author{Myriam B\'{e}gel}
\date{5 Septembre 2017}
%\institute{ENS Cachan, Computer Science department}
%\subtitle{Verification}
\bibliographystyle{apalike}

\newlength{\textlarg}

%Tikz
\usetikzlibrary{arrows,automata}
\usetikzlibrary{positioning}
\tikzset{
	state/.style={
		rectangle,
		rounded corners,
		very thick,
		minimum height=2em,
		inner sep=2pt,
		text centered,
node distance=1.3cm,
	},
}
\renewcommand{\ULthickness}{2pt}

\newcommand{\stkout}[1]{
	\textcolor{Data61 green}{
		\ifmmode\text{\sout{\ensuremath{#1}}}
		\else\sout{#1}\fi}}

\newcommand{\triple}[3]{\{#1\}~#2~\{#3\}}
\newcommand{\Dtriple}[3]{\vDash\{#1\}~#2~\{#3\}}
\newcommand{\ttriple}[3]{[#1]~#2~[#3]}
\newcommand{\tinst}[3]{{\color{Data61 green}\vdash_{inst}}\ttriple{#1}{#2}{#3}}
\newcommand{\tseq}[3]{{\color{Data61 plum}\vdash_{seq}}\ttriple{#1}{#2}{#3}}
\newcommand{\tblocks}[4]{#1{\color{Data61 ocean blue}\vdash_{blocks}}\ttriple{#2}{#3}{#4}}


\begin{document}
%%1
\maketitle


\begin{frame}{Vue d'ensemble}
	\centering
\begin{tikzpicture}[->,>=stealth']

% Position of QUERY 
% Use previously defined 'state' as layout (see above)
% use tabular for content to get columns/rows
% parbox to limit width of the listing
%Myreen's word

\node[state, 	% layout (defined above)
text width=10cm, 	% max text width
anchor=center,
draw=Data61 green] (Myreen) 
{\textcolor{Data61 green}{S\'{e}mantique pour code-machine \cite{Myreen09}}};

% Hirai's work
\uncover<2->{
\node[state,    	% layout (defined above)
text width=10cm, 	% max text width
below of=Myreen, 	% Position is to the right of QUERY
anchor=center,
draw=Data61 plum
] (Hirai) 	% posistion relative to the center of the 'box'
{%
\color{Data61 plum}Impl\'{e}mentation pour Ethereum \cite{hirai2017defining}
};
\path (Myreen) 	edge 	(Hirai);}

% Terminating program logic
\uncover<3->{
\node[state,
below of=Hirai,
text width=10cm, 	% max text width
anchor=center,
draw=Data61 vermillion
] (Terminating) 
{\color{Data61 vermillion}
	Correction totale pour Ethereum
};
\path (Hirai)     	edge 	(Terminating);}

% Inductive rules
\uncover<4->{
\node[state,
below of=Terminating,
text width=10cm, 	% max text width
anchor=center,
draw=Data61 ocean blue
] (Logic) 
{\color{Data61 ocean blue}
	Logique de programme correcte par rapport \`{a} la correction totale
};
\path (Terminating)	edge	(Logic);}

% Automation
\uncover<5->{
\node[state,
below of=Logic,
text width=10cm,	% max text width
anchor=center,
draw=Data61 ocean blue
] (Auto) 
{\color{Data61 ocean blue} Preuves automatiques et exemple
};
\path(Logic)       	edge  	(Auto)
;}

\end{tikzpicture}
\end{frame}

%%2
\part{Contexte}
\frame{\partpage}

%%3
\begin{frame}{Ethereum}
	\begin{columns}[c]
		\column{.6\textwidth}
		\begin{itemize}
			\item Une blockchain
			\item Bien plus que de la monnaie virtuelle
			\item \textit{Contrats intelligents} : programmes ex\'{e}cut\'{e}es par une machine virtuelle Turing compl\`{e}te
		\end{itemize}
		\column{.4\textwidth}
		\includegraphics[scale=0.3]{Figures/ETHEREUM-LOGO_PORTRAIT_Black_small.png}
	\end{columns}
\end{frame}

%%4
\begin{frame}{Machine Virtuelle d'Ethereum (EVM) \footnotesize{\cite{wood2014ethereum}}}
	\begin{columns}[c]
		\column{.6\textwidth}
		Quelques diff\'{e}rences avec JVM et ARM
		\begin{itemize}
			\item Terminaison par \textit{gas}
			\item JUMP dynamiques
		\end{itemize}
		\column{.4\textwidth}
		\begin{tabular}{r l}
			1 & \texttt{JUMPI} \\
			2 & \texttt{PUSH~20}\\
			4 & \texttt{PUSH~10}\\
			6 & \texttt{JUMP}\\
			7 & \texttt{JUMPDEST}\\
			8 & \texttt{PUSH~30}\\
			10 & \texttt{JUMPDEST}\\
			11 & \texttt{JUMP}
		\end{tabular}
	\end{columns}
\end{frame}

%%5
\begin{frame}{But = contrats fiables}
	La plupart des contrats sont critiques (finance, propri\'{e}t\'{e}, ...)\\
	$\Rightarrow$ contribution \`{a} un projet open-source pour formellement sp\'{e}cifier et v\'{e}rifier les contrats
	\\ ~ \\
	\textbf{\'{E}tat de l'art} \\
	EVM d\'{e}fini en \textit{Isabelle} \cite{hirai2017defining} et une logique de Hoare inspir\'{e}e de \cite{Myreen09}
\end{frame}

%%6
\begin{frame}{V\'{e}rification}
	Au niveau du bytecode
	\begin{itemize}
		\item Garanties fortes
		\item Pas de trou dans la cha\^{i}ne de confiance
%		\item Point faible actuel = Solidity
		\item Plusieurs langages de programmation
	\end{itemize}
\end{frame}

%%7
\begin{frame}{Probl\`{e}me}
	\only<1-8>{\textbf{Correction partielle}\\
	$\vDash\{P\}c\{Q\}$ si $Q$ valide \alert<1,6-7>{\`{a} un certain moment} dans l'ex\'{e}cution de $c$}
	
	\only<9->{\textbf{Correction \textcolor{Data61 green}{totale}}\\
	${\color{Data61 dark mint}\vDash}[P]c[Q]$ si $Q$ valide {\color{Data61 green}\`{a} la fin} de l'ex\'{e}cution de $c$}

	\begin{align*}
	\only<1-8>{\vDash&\{P\}c\{Q\}\equiv\\}
	\only<9>{{\color{Data61 dark mint}\vDash~}&[P]c[Q]\equiv\\}
	&~\forall cctx~ ir~ R.\\
	&\quad\alert<4>{(P \alert<2>{\wedge^*} \t{code } c \alert<2>{\wedge^*} R)}\\
	&\qquad\alert<4>{(\t{inst\_result\_as\_set } \alert<3>{cctx~ ir})} \Rightarrow\\
	&\quad
	\only<1-8>{~\alert<7-8>{\exists k.}}
	\only<9>{\stkout{\exists k.}}
	\alert<6>{(Q \alert<2>{\wedge^*} \t{code }c \alert<2>{\wedge^*} R)}\\
	&\qquad\alert<6>{(\t{inst\_result\_as\_set }cctx}~
	\only<1-8>{\alert<5-6>{(\t{run }~~ cctx~ \alert<7-8>{k}~ir))}}
	\only<9>{(\t{run\textcolor{Data61 green}{\_t} } cctx~ \stkout{k}~ir))}
	\end{align*}

	\uncover<8>{\textbf{On peut mieux faire}\\}
	\uncover<9>{Terminaison de $\texttt{run\_t}$ prouv\'{e}es par mes coll\`{e}gues}
\end{frame}

%%8
\part{Contributions}
\frame{\partpage}

%%9
\begin{frame}{Logique de programme}
	3 niveaux pour la logique \\
%	\begin{columns}
%		\column{.4\textwidth}
	\begin{itemize}
		\item<2-> Instructions ${\color{Data61 green}\vdash_{inst}}$
			\begin{itemize}
				\item une r\`{e}gle par instruction
%				\item<5-> non-terminating semantic with 1 step
			\end{itemize}
		\item<3-> Un \textit{bloc s\'{e}quentiel} ${\color{Data61 plum}\vdash_{seq}}$
			\begin{itemize}
				\item d\'{e}rouler une liste d'instruction
%				\item<5-> non-terminating semantic with $n$ steps \\ $n=$ number of instructions
			\end{itemize}
		\item<4-> Connecter les blocs ${\color{Data61 ocean blue}\vdash_{blocks}}$
			\begin{itemize}
				\item aller d'un bloc au suivant
%				\item<6-> terminating semantic
			\end{itemize}
	\end{itemize}
	\uncover<5>{
	\begin{figure}
		\[
		\dfrac
		{\dfrac
			{\dfrac{}{\tinst{P}{hd~b}{R_1}}
				\quad
				\dfrac
				{\dfrac{}{{\color{Data61 green}\vdash_{inst}} \sim} \quad
					\dfrac{\dots}{{\color{Data61 plum}\vdash_{seq}} \sim}}
				{\tseq{R_1}{tl~b}{R}}}
			{\tseq{P}{b}{R}}
			\quad \dfrac{\dots}{\tblocks{blocks}{R}{next~block}{Q}}}
		{\tblocks{blocks}{P}{b}{Q}}\]
	\end{figure}}
\end{frame}

%%10
\begin{frame}{Bloc s\'{e}quentiel}
	\uncover<4->{R\'{e}sultat de $\t{build\_blocks}$:\\}
	\centering
	\only<-2>{
	\begin{tabular}{c r l c}
		&	{0}	&	OR&\\
		&	{1}	&	ADD&\\
		\uncover<4->{0}&	{2}	&	SWAP1& 	\uncover<4->{Next}\\
		&	{3}	&	\color<2->{Data61 green}JUMPDEST&\\
		&	{4}	&	MLOAD&\\
		&	{5}	&	POP&\\
		\uncover<4->{3}&	{6}	&	\color<2->{Data61 green}JUMP& 	\uncover<4->{Jump}\\
		&	{7}	&	DUP3&\\
		&	{8}	&	PUSH 0&\\
		&	{10}	&	ISZERO&\\
		\uncover<4->{7}&	{11}	&	\color<2->{Data61 green}JUMPI& 	\uncover<4->{Jumpi}\\
		&	{12}	&	POP&\\
		\uncover<4->{11}&	{13}	&	\color<2->{Data61 green}RETURN& 	\uncover<4->{Terminal}\\
		\uncover<4->{14}&	{14}	&	AND	& 	\uncover<4->{Next}
	\end{tabular}}
\setbeamercovered{transparent}
	\only<3->{
	\begin{tabular}{c r l c}
		&	0	&	OR&\\
		&	1	&	ADD&\\
		\visible<4->{0}&	2	&	SWAP1& 	\visible<4->{Next}\\
			\hline
		&	3	&	\color{Data61 green}JUMPDEST&\\
		&	4	&	MLOAD&\\
		&	5	&	POP&\\
		\visible<4->{3}&	\uncover<-3>{6}	&	\uncover<-3>{\color{Data61 green}JUMP}&	\visible<4->{Jump}\\
			\hline
		&	7	&	DUP3&\\
		&	8	&	PUSH 0&\\
		&	10	&	ISZERO&\\
		\visible<4->{7}&	\uncover<-3>{11}	&	\uncover<-3>{\color{Data61 green}JUMPI}&	\visible<4->{Jumpi}\\
			\hline
		&	12	&	POP&\\
		\visible<4->{12}&	13	&	\color{Data61 green}RETURN&	\visible<4->{Terminal}\\
			\hline
		\visible<4->{14}&	14	&	AND	& 	\visible<4->{Next}
	\end{tabular}}
\end{frame}
\setbeamercovered{}
%%11
\begin{frame}{R\`{e}gle pour POP}
	\begin{align*}
		{\color{Data61 green}\vdash_{inst}}&\\
		[&{\color<2>{Data61 plum}\langle h \leq 1023 \wedge 2 \leq g \rangle}~ \wedge^*\\
		&{\color<3>{Data61 dark mint}\texttt{continuing }} \wedge^*
		{\color<4>{Data61 vermillion}\texttt{program-counter n }} \wedge^*
		{\color<5>{Data61 ocean blue}\texttt{gas-pred g }} \wedge^*\\
		&{\color<6>{Data61 fuchsia}\texttt{stack-height (Suc h) }} \wedge^*
		{\color<6>{Data61 fuchsia}\texttt{stack h x }} \wedge^*
		{\color<7>{Data61 light forest}\texttt{rest}}
		\\
		(&\texttt{\textcolor<4>{Data61 vermillion}{n}, \textcolor<1>{Data61 light teal}{POP}}) \\
		[&{\color<3>{Data61 dark mint}\texttt{continuing }} \wedge^*
		{\color<4>{Data61 vermillion}\texttt{program-counter (n + 1) }} \wedge^*\\
		&{\color<5>{Data61 ocean blue}\texttt{gas-pred (g - 2) }} \wedge^*
		{\color<6>{Data61 fuchsia}\texttt{stack-height h }} \wedge^*
		{\color<7>{Data61 light forest}\texttt{rest}}]
	\end{align*}
\end{frame}

%%12
\begin{frame}{R\`{e}gles pour un bloc s\'{e}quentiel}
	\[\frac{{\color{Data61 green}\vdash_{inst}} [P] ~x~ [R]
		\qquad {\color{Data61 plum}\vdash_{seq}} [R] ~xs~ [Q]}
	{{\color{Data61 plum}\vdash_{seq}} [P] ~x\#xs~ [Q]}\]

	\[\frac{\forall s.~P~s~\Rightarrow~Q~s}
	{{\color{Data61 plum}\vdash_{seq}} [P] ~\t{Nil}~ [Q]}\]
\end{frame}

%%13
\begin{frame}{R\`{e}gles pour les blocs Terminal et Next}
	\[\frac{{\color{Data61 plum}\vdash_{seq}} [P] xs [Q]}
	{\texttt{blocks~}{\color{Data61 ocean blue}\vdash_{blocks}} [P] (n,xs,Terminal) [Q]}\]

	~\\~\\
	\centering
	$\frac{{\color{Data61 plum}\vdash_{seq}} [P] xs [\texttt{pc } m \wedge^* R]
		\quad \texttt{(m,ys,t)}\in \texttt{blocks}
		\quad \texttt{blocks~}{\color{Data61 ocean blue}\vdash_{blocks}} [\texttt{pc } m \wedge^* R] (m,ys,t) [Q]}
	{\texttt{blocks~}{\color{Data61 ocean blue}\vdash_{blocks}} [P] (n,xs,Next) [Q]}$
\end{frame}

%%14
\begin{frame}{R\`{e}gle pour un bloc Jump}
	\footnotesize
	$$\frac{{\color{Data61 plum}\vdash_{seq}} [P] xs [R1]
	\quad \begin{aligned}
	&\texttt{(m,ys,t)}\in \texttt{blocks}\\
	&\texttt{hd ys = (m, JUMPDEST)}\\
	\end{aligned}
	\quad \texttt{blocks~} {\color{Data61 ocean blue}\vdash_{blocks}} [R2] (m,ys,t) [Q]}
	{\texttt{blocks~}{\color{Data61 ocean blue}\vdash_{blocks}} [P] (n,xs,Jump) [Q]}$$
	\small
	\uncover<2>{
	\begin{align*}
	R1 = ~& {\color{Data61 plum}\langle h \leq 1023 \wedge \color{Data61 plum}8 \leq g \rangle}~ \wedge^* 
	{\color{Data61 dark mint}\texttt{continuing }} \wedge^*
	{\color{Data61 ocean blue}\texttt{gas-pred g }} \wedge^*\\
	&{\color{Data61 vermillion}\texttt{program-counter p }}\wedge^*\\
	&{\color{Data61 fuchsia}\texttt{stack-height (Suc h) }} \wedge^*
	{\color{Data61 fuchsia}\texttt{stack h m }} \wedge^*
	{\color{Data61 light forest}\texttt{rest}}\\
	\end{align*}
	\begin{align*}
	R2 = ~&{\color{Data61 dark mint}\texttt{continuing }} \wedge^*
	{\color{Data61 ocean blue}\texttt{gas-pred (g - 8) }} \wedge^*\\
	&{\color{Data61 vermillion}\texttt{program-counter m }}\wedge^*
	{\color{Data61 fuchsia}\texttt{stack-height h }} \wedge^*
	{\color{Data61 light forest}\texttt{rest}}\\
	\end{align*}
	}
\end{frame}

%%15
\begin{frame}{R\`{e}gle pour un bloc Jumpi}
	\footnotesize
	\hspace*{-20pt}
	$\tfrac{
		\begin{aligned}
		\texttt{hd }&\texttt{ys = (m, JUMPDEST)}&\\
		& \texttt{(m,ys,t)}\in \texttt{bl}
		&c \neq 0 \Rightarrow \texttt{bl}{\color{Data61 ocean blue}\vdash_{blocks}} [\textcolor<2>{Data61 vermillion}{\t{pc } m} \wedge^* R2] (m,ys,t) [Q]
		\\
		{\color{Data61 plum}\vdash_{seq}} [P] xs [R1] \quad
		&\texttt{(p,zs,t')}\in \texttt{bl}
		&c=0 \Rightarrow\texttt{bl}{\color{Data61 ocean blue}\vdash_{blocks}} [\textcolor<2>{Data61 vermillion}{\t{pc } p} \wedge^* R2] (p,zs,t') [Q]
		\end{aligned}
		}
	{\begin{aligned}\texttt{bl~}{\color{Data61 ocean blue}\vdash_{blocks}} [P] (n,xs,Jumpi) [Q]\end{aligned}}$
	\
	\uncover<2>{
	\begin{align*}
	R1 = ~& {\color{Data61 plum}\langle h \leq 1022 \wedge \color{Data61 plum}10 \leq g \rangle}~ \wedge^*
	{\color{Data61 dark mint}\texttt{continuing }} \wedge^*\\
	&{\color{Data61 vermillion}\texttt{program-counter (p - 1) }}\wedge^*
	{\color{Data61 ocean blue}\texttt{gas-pred g }} \wedge^*\\
	&{\color{Data61 fuchsia}\texttt{stack-height (h + 2) }} \wedge^*
	{\color{Data61 fuchsia}\texttt{stack (h+1) m }} \wedge^*
	{\color{Data61 fuchsia}\texttt{stack h c }} \wedge^*
	{\color{Data61 light forest}\texttt{rest}}\\
	\end{align*}
	\begin{align*}
	R2 = ~&{\color{Data61 dark mint}\texttt{continuing }} \wedge^*
	{\color{Data61 ocean blue}\texttt{gas-pred (g - 10) }} \wedge^*\\
	&{\color{Data61 fuchsia}\texttt{stack-height h }} \wedge^*
	{\color{Data61 light forest}\texttt{rest}}\\
	\end{align*}
	}
\end{frame}

%%16
\begin{frame}{Correction}
	\[{\color{Data61 green}\vdash_{inst}}~[P]~i~[Q]~ \Rightarrow {\color{Data61 dark mint}\vDash} [P]~\{i\}~[Q]\]
	\uncover<2->{\[{\color{Data61 plum}\vdash_{seq}}~[P]~xs~[Q]~ \Rightarrow {\color{Data61 dark mint}\vDash} [P]~(\t{set } xs)~[Q]\]}
	\uncover<3->{
	\begin{gather*}
		\uncover<4>{0<  \vert c\vert < 2^{256} \Rightarrow\\}
		{\t{build\_blocks }c~{\color{Data61 ocean blue}\vdash_\t{blocks}} [P]~(fst\_block)~[Q]}\Rightarrow\\
		{\color{Data61 dark mint}\vDash} [P]~(\t{set } c)~[Q]
	\end{gather*}}
%	${\color{Data61 dark mint}\vDash}$ = terminating semantic for bytecode
\end{frame}

\begin{frame}{V\'{e}rification d'un contrat simple}
		\footnotesize
	\begin{columns}[c]
		\column{.46\textwidth}
		Contrat en \textit{Solidity} :\\
		\ttfamily
		\textcolor{Data61 ocean blue}{contract} D \{ \\
		\quad	\textcolor{Data61 ocean blue}{function} dispatch1()\\
		\quad	\textcolor{Data61 ocean blue}{returns} (\textcolor{Data61 midnight blue}{uint}) \{\\
		\quad\quad	return (1);\\
		\quad	\}\\
		\quad	\textcolor{Data61 ocean blue}{function} dispatch2()\\
		\quad	\textcolor{Data61 ocean blue}{returns} (\textcolor{Data61 midnight blue}{uint}) \{\\
		\quad\quad	{return} (2);\\
		\quad	\}\\
		\}\\
		
		\column{.6\textwidth}
		\uncover<3->{Sp\'{e}cification en Isabelle :\\
		\ttfamily
		\textcolor{Data61 dark mint}{definition}\\
		\quad return$\_$action\\
		\textcolor{Data61 dark mint}{where}\\
		"return$\_$action hash = \\
		(\alert<3>{\textcolor{Data61 dark mint}{if} hash = dispatch1$\_$hash}\\
		\alert<3>{\textcolor{Data61 dark mint}{then} ContractReturn 1}\\
		\textcolor{Data61 dark mint}{else}\\
		(\alert<4>{\textcolor{Data61 dark mint}{if} hash = dispatch2$\_$hash}\\
		\alert<4>{\textcolor{Data61 dark mint}{then} ContractReturn 2}\\
		\alert<5>{\textcolor{Data61 dark mint}{else} ContractFail [ShouldNotHappen]}))"}
	\end{columns}
~\\~\\
\uncover<2->{Invoquer un contrat = appel d'une fonction en particulier}\\
\uncover<3->{Dispatcher v\'{e}rifiable uniquement \`{a} partir du bytecode}
\end{frame}

\begin{frame}{Preuve automatique}
	\footnotesize
	{\ttfamily
	\textcolor{Data61 dark mint}{lemma}:\\
	"$\exists$ r.\\
	build$\_$blocks D ${\color{Data61 ocean blue}\vdash_\t{blocks}}$ \\
	$[$program$\_$counter 0 $\wedge^*$ stack$\_$height 0 $\wedge^*$ sent$\_$data hash $\wedge^*$ sent$\_$value 0 $\wedge^*$
	memory$\_$usage 0 $\wedge^*$ continuing $\wedge^*$ gas$\_$pred 3000 $\wedge^*$ memory 64 x $\wedge^*$
	memory 96 y]\\
	fst$\_$block D\\
	$[$action (return$\_$action hash) $\wedge^*$ r]"}
	
	~\\
	\normalsize
	\uncover<2>{
	Contrat de 226 octets \\
	14 lignes d'Isabelle gr\^{a}ce \`{a} l'automatisation}
\end{frame}


%%19
\part{Travail \`{a} venir}
\frame{\partpage}

%%17
\begin{frame}{R\'{e}sum\'{e} - contributions}
	\begin{itemize}
		\item D\'{e}finir la d\'{e}coupe du bytecode en bloc en Isabelle
		\item Prouver la d\'{e}coupe correcte
		\item D\'{e}finir une logique de programme pour la correction totale
		\item Prouver la correction de la logique
		\item Automatiser les preuves
	\end{itemize}
	
\end{frame}


%%20
\begin{frame}{Sp\'{e}cification d'autres contrats}
	\begin{itemize}
		\item \'{E}crire les r\`{e}gles pour toutes les instructions (31/70)
		\item Les prouver correctes
		\item Sp\'{e}cification de contrat plus complexes
	\end{itemize}
\end{frame}

%%21
%\begin{frame}{Control flow graph}
%	CFG was the initial goal\\
%	BUT problems with dynamic jumps\\
%	HOWEVER proposition of static jumps and explicit function calls for EVM 1.5
%	\\~\\
%	\begin{itemize}
%		\item Add edges to our blocks
%		\item Detect loops
%		\item More structured program
%	\end{itemize}
%\end{frame}

%\begin{frame}{My internship}
%	\setbeamercovered{transparent}
%		\begin{itemize}
%		\item Learning more about blockchain
%		\item<2-> Discovering Isabelle
%		\item<3-> Love formal verification again
%		\item<4-> Research is not working all alone
%		\item<5-> \alert<5>{This group is great}
%	\end{itemize}
%\end{frame}

%%22
\frame[plain]{\thankspage{}{}{}{myriam.begel@ens-paris-saclay.fr}{}}

% \part \partpage not implemented yet
% thanks/question page not implemented

\bibliography{../bibli}

\end{document}